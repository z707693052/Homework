\documentclass[10pt,a4paper,nocap]{ctexart}
\usepackage{amsmath,amssymb,amsthm}
\usepackage{graphicx}
\usepackage{environ}
\usepackage{multirow}
\DeclareGraphicsExtensions{.pdf,.png,.jpg,.eps}
\graphicspath{{C:/Users/cengqQ/Pictures/}}
\linespread{1.5}
\newcommand{\dd}[0]{\mathrm{d}}
\NewEnviron{formula}{ 
\begin{flalign*}\begin{split}
\BODY
\end{split}&\end{flalign*}}
\begin{document}
	\title{光栅组合实验报告}
	\author{基科32 曾柯又 2013012266}
	\date{\vspace{-5ex}}
	\maketitle
	\CTEXsetup[format={\centering}]{part}
	\CTEXsetup[titleformat={\large}]{section}
	\CTEXsetup[nameformat = {\Large\bfseries}]{part}
	\CTEXsetup[titleformat = {\Large\bfseries}]{part}
	\CTEXsetup[titleformat = {\normalsize }]{subsection}
	\part{基础实验部分}
	\section{实验目的}
	\subparagraph{(1)}根据已知波长的谱线,利用正入射,斜入射法测量光栅的光栅常量。
	\subparagraph{(2)}用已知波长的He-Ne光谱作参考,测量氢原子较强可见光谱线的波长,计算氢的里德伯常量。
	\section{实验原理}
		\subsection{平面光栅的衍射}
	平面反射光栅是刻有一系列等间距平行划痕的反射平面镜,设光栅常量为$ d $,则可以得到出射光束相干涉出现极大值的条件为:\[d(\sin\theta_{rk} - \sin\theta_i) = d(\sin(\phi_n - \phi_{rk}) - \sin(\phi_n - \phi_i))= K\lambda,\; K = 0,\pm1,\pm2 \cdots\]
	
	 式中,\(\phi_n\)为光栅表面方位角,\(\theta_i = \phi
		 _n - \phi_i\)为入射角,\(\theta_{rk} = \phi_n - \phi_{rk}\)为衍射角,\(\lambda\)是波长,于是在满足上式的一系列出射角\(\theta_{rk}\)的方向上将观察到亮谱线。实验中的测量一般采用两种方式。
		 \paragraph{(1)}正入射法
		 
		 对于入射角\(\theta_i = 0\),衍射公式简化为:\[d\sin\theta_{rk} = K\lambda\]
		 \paragraph{(2)}斜入射法
		 
		 即对应入射角\(\theta_i \neq 0\) ,衍射公式仍为:\[d(\sin\theta_{rk} - \sin\theta_i) = K\lambda \]
		 
		 利用上述两式,测得不同级次下已知波长谱线的衍射角,即可求得光栅的光栅常量\(d\)。
		\section{巴尔末公式与里德伯常量}
		巴尔末线系4条可见光谱线的经验公式可以写为:\(\displaystyle \lambda_0 = R_H^{-1}\left(\frac{1}{2^2} - \frac{1}{N^2}\right)\),式中\(\lambda_0\)是真空波长,\(R_H\)是氢的里德伯常量。又由波尔的量子论可得:\[\lambda_0 = R_{\infty}^{-1}(1 + \frac{m_e}{m_p})(\frac{1}{2^2} - \frac{1}{N^2})^{-1}\]
		上式中,\(R_\infty\)是里德伯常量,\(m_e\)为电子质量,\(m_p\)为质子质量,在考虑到空气折射率的影响后,可由下式计算\(R_\infty\):
		\[R_\infty \approx (1.00028\lambda)^{-1}(1 + \frac{m_e}{m_p})\left(\frac{1}{2^2} - \frac{1}{N^2}\right)\]
		
		在实验中,测定氦氖谱线和氢的谱线的衍射方位角,利用已知氦氖谱线波长,求得一次回归方程:\[\phi =b_0 + b_1\lambda\]
		或二次回归方程\[\phi = b_0 + b_1\lambda + b_2\lambda^2\]
		
		将氢的衍射角\(\phi_H\)带入回归方程,即可求得氢红线谱线波长\(\lambda_H\),进而可以求得里德伯常量。
		\section{实验数据及处理}
		\subsection{正入射法}\begin{center}
			
		\begin{tabular}{ |c |c| c| c| c| }\hline
		 \multicolumn{2}{|c|}{} & K = 1	& K = 2	& K = 3\\ \hline
		蓝紫&\(\phi_1/\textdegree\) & 7.52  & 15.15  & 23.08 \\ \cline{2-5}
		\(\lambda = 435.83\) &\(\phi_2/\textdegree\) &  7.52 & 15.15 & 23.08\\ \hline 
		 \multicolumn{2}{|c|}{\(d = \frac{K\lambda}{\sin\phi}\)} &3331.6621& 3335.2599& 3334.8421\\ \hline
		黄绿&\(\phi_1/\textdegree\) &  9.43 & 19.13 & 29.44 \\ \cline{2-5}
		 \(\lambda = 546.07\)& \(\phi_2/\textdegree\) & 9.43 & 19.13 & 29.44 \\ \hline
		 \multicolumn{2}{|c|}{\(d = \frac{K\lambda}{\sin\phi}\)} & 3331.7287 & 3333.4535 & 3332.8302\\ \hline
		 \end{tabular}\\
		\end{center}
		平均值为:
		\[\bar{d} = 3333.296125 \mathrm{nm} = 3.333296 \times10^{-6} \mathrm{m}\]
		\subsection{斜入射法}
		以\(15\textdegree\)斜入射
	\begin{center}
			\begin{tabular}{ |c |c| c| c| c| }\hline
			 \multicolumn{2}{|c|}{} & K = 1	& K = 2	& K = 3\\ \hline
			蓝紫&\(\phi_1/\textdegree\) & -7.3583	& 0.15 & -7.625  \\ \cline{2-5}
		\(\lambda = 435.83\) &\(\phi_2/\textdegree\) &-7.3583 & 0.15 & 7.625 \\ \hline 
		\multicolumn{2}{|c|}{\(d\)} & 3333.4443 & 3334.1106 & 3339.6259 \\ \hline
		黄绿&\(\phi_1/\textdegree\) & -5.4583	& 3.925 & 13.425  \\ \cline{2-5}
		 \(\lambda = 546.07\)& \(\phi_2/\textdegree\) &-5.4583 & 3.925 & 13.425 \\ \hline 
		\multicolumn{2}{|c|}{\(d\)} & 3335.8544 & 3337.1258 & 3336.5351 \\ \hline
		\end{tabular}\\
	\end{center}
		平均值为:
			\[\bar{d} = 3336.1160 \mathrm{nm} = 3.336116 \times10^{-6} \mathrm{m}\]
	\subsection{ 不同级次谱线测量的误差讨论}
			正入射情况下:\[d = \frac{K\lambda}{\sin\phi}\]
			可以得到:\[\frac{\Delta d}{d} = \frac{\Delta\phi}{\phi}\]
			
			可以看出,测量d时,当\(\Delta\phi\)一定时,\(\phi\)越大,\(\Delta d\)越小,因此测量的级次越高,\(\Delta d\)越小。但是衍射的级次越高,谱线的形变越厉害,并且级次越高,谱线也越难以观察,因此需要综合考虑。
	\subsection{比较法测里德伯常量}
	\begin{center}
		\begin{tabular}{|c|c|c|}\hline
			\(\lambda/nm\)	& \(\lambda^2/nm\) & \(\phi/\textdegree\)	\\ \hline
			638.2992 & 407425.8687 & 184.95	\\ \hline
			640.2246 & 409887.5384 & 185.0917 \\ \hline
			650.6528 & 423349.0661 & 185.8083 \\ \hline
			653.2882 & 426785.4723 & 185.9917 \\ \hline
			659.8953 & 435461.807 &	186.4417 \\ \hline
			667.815	 & 445976.8742 & 186.9917 \\ \hline
			671.743	 & 451238.658 & 187.25 \\ \hline
	\end{tabular}
	\end{center}
	\( \phi_H = 186.2\textdegree \)
	\subsubsection{一次拟合结果}
\begin{center}
		\begin{tabular}{|c|c|c|c|}\hline
		\(b_0\)& \(s_{b_0}\) & \(b_1\) & \(s_{b_1}\) \\ \hline
		 141.0527 & 0.1117 & 0.0688  & \(1.7067\times10^{-4}\) \\ \hline
	\end{tabular}\\
\end{center}
	 因变量标准差\(s_\phi =0.00536\textdegree\)\\
	 可以解出\(\lambda_H = 656.3771nm\)\\
	 进而求得:\(R_\infty = 1097220.676 \mathrm{m}^{-1}\)
	 
\subsubsection{二次拟合结果}
\begin{center}
	\begin{tabular}{|c|c|c|c|c|c|} \hline
		\(b_0\) & \(s_{b_0}\) & \(b_1\) & \(s_{b_1}\) &\(b_2\) & \(s_{b_2}\) \\ \hline
		130.2325 & 6.4981 & 0.1018 & 0.0198 & \(-2.5253\times10^{-5}\) & \(1.5164\times10^{-5}\)\\ \hline
	\end{tabular}
\end{center}
因变量标准差\(s_\phi = 0.0046\textdegree\)\\
解出\(\lambda_H = 656.3263\mathrm{nm}\)\\
进而解得\(R_\infty = 1097305.506\mathrm{m}^{-1}\)

可以发现直线拟合的\(s_\phi\)和二次拟合的\(s_\phi\)相差不大,\(\displaystyle \frac{|s_{\phi1} - s_{\phi2}|}{s_{\phi2}} = 10.56 \%\),并且结果也相差不大,并且直线拟合更加简便,因此一般用直线拟合就足够了。
\part{探究性实验部分}
\section{实验目的}
我选择的实验是氢原子光谱及其的同位素位移的观测。即利用长焦数码相机成像的方法,计算与测量氢氘混合气体放电管的红光光谱波长差,研究验证氢原子光谱的同位素位移率。
\section{实验原理}
光栅衍射的基本原理同前基础实验部分,这里主要补充氢原子光谱的同位素位移率
\subsection{}
\end{document}
