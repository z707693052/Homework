\documentclass[10pt,a4paper,nocap]{ctexart}
\usepackage{amsmath,amssymb,amsthm}
\usepackage{graphicx}
\usepackage{environ}
\DeclareGraphicsExtensions{.pdf,.png,.jpg,.eps}
\graphicspath{{C:/Users/cengqQ/Pictures/}}
\linespread{1.5}
\newcommand{\dd}[0]{\mathrm{d}}
\NewEnviron{formula}{ 
\begin{flalign*}\begin{split}
\BODY
\end{split}&\end{flalign*}}
\begin{document}
	\title{光栅组合实验报告}
	\author{基科32 曾柯又 2013012266}
	\date{\vspace{-5ex}}
	\maketitle
	\CTEXsetup[format={\centering}]{part}
	\CTEXsetup[titleformat={\large}]{section}
	\CTEXsetup[nameformat = {\Large\bfseries}]{part}
	\CTEXsetup[titleformat = {\Large\bfseries}]{part}
	\part{基础实验部分}
	\section{实验目的}
	\subparagraph{(1)}根据已知波长的谱线,利用正入射,斜入射法测量光栅的光栅常量。
	\subparagraph{(2)}用已知波长的He-Ne光谱作参考,测量氢原子较强可见光谱线的波长,计算氢的里德伯常量。
	\section{实验原理}
	\subsection{平面光栅的衍射}
	平面反射光栅是刻有一系列等间距平行划痕的反射平面镜,设光栅常量为$ d $,则可以得到出射光束相干涉出现极大值的条件为:\[d(\sin\theta_{rk} - \sin\theta_i) = d(\sin(\phi_n - \phi_{rk}) - \sin(\phi_n - \phi_i))= K\lambda,\; K = 0,\pm1,\pm2 \cdots\]
	
	 式中,\(\phi_n\)为光栅表面方位角,\(\theta_i = \phi
		 _n - \phi_i\)为入射角,\(\theta_{rk} = \phi_n - \phi_{rk}\)为衍射角,\(\lambda\)是波长,于是在满足上式的一系列出射角\(\theta_{rk}\)的方向上将观察到亮谱线。实验中的测量一般采用两种方式。
		 \paragraph{(1)}正入射法
		 
		 对于入射角\(\theta_i = 0\),衍射公式简化为:\[d\sin\theta_{rk} = K\lambda\]
		 \paragraph{(2)}斜入射法
		 
		 即对应入射角\(\theta_i \neq 0\) ,衍射公式仍为:\[d(\sin\theta_{rk} - \sin\theta_i) = K\lambda \]
		 
		 利用上述两式,测得不同级次下已知波长谱线的衍射角,即可求得光栅的光栅常量\(d\)。
		\section{巴尔末公式与里德伯常量}
		巴尔末纤细4条可见光谱线的经验公式可以写为,
\end{document}
