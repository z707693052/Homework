\documentclass[a4paper]{ctexart}
\usepackage{amsmath,amssymb,amsthm}
\begin{document}
\title{}
\author{基科32 曾柯又 2013012266}
\date{}
\maketitle
\paragraph{1}作长除法
\newdimen\digitwidth
\settowidth\digitwidth{0}
\def~{\hspace{\digitwidth}}
\def\divrule#1#2{ %
\noalign{\moveright#1\digitwidth%
\vbox{\hrule width#2\digitwidth}}}
110101\,
\begin{tabular}[b]{@{}r@{}}
1101010110 \\ \hline
\big) \begin{tabular}[t]{@{}l@{}}
101000110100000\\
110101\\ \divrule{0}{7}
~111011\\
~110101\\ \divrule{1}{8}
~~0111010\\
~~~110101\\ \divrule{3}{8}
~~~~0111110\\
~~~~~110101 \\ \divrule{5}{8}
~~~~~~0101100\\
~~~~~~~110101\\ \divrule{7}{7}
~~~~~~~0110010\\
~~~~~~~~110101\\ \divrule{9}{6}
~~~~~~~~~001110\\
~~~~~~~~~000000\\ \divrule{9}{6}
~~~~~~~~~~~1110
\end{tabular}
\end{tabular}\\
得到余数1110,因此发CRC校验码为101000110101110。
\paragraph{2}
通过计算可以得到1010001101的汉明码是00110101001101\\
如果接收到的汉明码是00100100001101,可计算得4,8位校验码出错,因此12位出错,应变为0,则得原始数据比特列为1010000101
\end{document}
