\documentclass[11pt,a4paper]{ctexart}
\usepackage{amsmath,amssymb,amsthm}
\usepackage{graphicx}
\DeclareGraphicsExtensions{.pdf,.png,.jpg,.eps}
\graphicspath{{/home/cengq/Pictures/}}
\linespread{1.9}
\CTEXsetup[indent={0pt}]{subparagraph}
\newcommand{\normal}[2]{\frac{1}{\sqrt{2\pi}#2}e^{-(x - #1)^2/(2#2^2)}}
\newcommand{\norm}[1]{\frac{1}{\sqrt{2\pi}#1}e^{-x^2/(2#1^2)}}
\newcommand{\norms}[0]{\frac{1}{\sqrt{2\pi}}e^{-x^2/2}}
\newcommand{\dconverge}[0]{\overset{\mathcal{D}}{\to}}
\newcommand{\pconverge}[0]{\overset{\mathcal{P}}{\to}}
\newcommand{\dd}[0]{\mathrm{d}}
\title{\vspace{-5ex}}
\author{基科32 曾柯又 2013012266}
\date{\vspace{-5ex}}
\begin{document}
\abovedisplayskip=5pt
\belowdisplayskip=5pt
\abovedisplayshortskip=0pt
\belowdisplayshortskip=0pt
\maketitle
\paragraph{模拟作业}
\subparagraph{[1]}
由\(\displaystyle \int_{0}^{2}g(x)\mathrm{d}x = \frac{3}{2}\)故令\(K = \frac{2}{3}\),此时\(f(x) = Kg(x)\)为密度函数
\subparagraph{[2]}
\(0 < x < 0.5\)时
\[F(x) = \int_{0}^{x}\frac{2}{3}2x\dd x = \frac{2x^2}{3} \]
\(0.5 < x < 1\)时
\[F(x) = \frac{1}{6} + \int_{0.5}^{x}\frac{2}{3}(2 - 2x)\dd x = \frac{1}{3} - \frac{2}{3}(x - 1)^2\]
\(1 < x < 1.5\)时
\[F(x) = \frac{1}{3} + \int_{0}^{x - 1}\frac{2}{3}4x\dd x = \frac{1}{3} + \frac{4}{3}(x - 1)^2\]
\(1.5 < x < 2\)时
\[F(x) = \frac{2}{3} + \int_{1.5}^{x}\frac{2}{3}(8 - 4x)\dd x = 1 - \frac{4}{3}(x - 2)^2  \]
做出\(f(x),F(x)\)的图形如下:\\
\includegraphics[scale=0.5]{Rf}\includegraphics[scale=0.5]{RF}\\
计算\(\displaystyle \mu = EX = \frac{7}{6}\) , \(\displaystyle\sigma^2 = VarX = \frac{19}{72} \)
\subparagraph{[3]}
利用直接方法产生随机数:\\
\indent 用分段函数表示\(F^{-1}(x)\),作用于产生的1000个(0,1)均匀分布随即变量,得到的1000个数即满足\(f(x)\)的分布,计算所得的\(\hat{\mu}_1 = 1.179\)
\subparagraph{[4]}
利用接收-拒绝法产生随机数:\\
\indent 这里我为了避免使用循环语句,而尽可能向量化,采取了先生成两列足够多的均匀分布随机数列,用向量操作筛选出满足\(y < f(x)\)数据的方法,但是由于向量长度是固定的,还要提前取定足够的数据使得筛选后至少有1000个数。这可以估计如下,满足要求的数分布在\(f(x)\)下方,面积为1,而所有数据在大方框内,满足要求概率为\(\frac{3}{8}\),要得到1000个数据,可以先生成2800组随即数,即基本能保证数据量,这样做虽然会额外生成一些数,但是在循环次数更多的时候更能节省时间。\\
两种方法获得直方图与概率密度函数叠加所得图如下\\
{\centering \includegraphics[scale = 0.5]{Rdirect} \includegraphics[scale=0.5]{Raccept}}
\subparagraph{[5]}
两种方法各重复2000次:
由于每次重复的步骤都是一样的,我并没有写成循环,而是直接产生\(1000\times2000\)组数据,再分成2000组,分组求平均,这样与循环2000次是一样的,速度也要快一点,
\end{document}
