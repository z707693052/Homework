\documentclass[11pt,a4paper]{ctexart}
\usepackage{amsmath,amssymb,amsthm}
\usepackage{graphicx}
\DeclareGraphicsExtensions{.pdf,.png,.jpg,.eps}
\graphicspath{{/home/cengq/Pictures/}}
\linespread{1.9}
\CTEXsetup[indent={0pt}]{subparagraph}
\newcommand{\normal}[2]{\frac{1}{\sqrt{2\pi}#2}e^{-(x - #1)^2/(2#2^2)}}
\newcommand{\norm}[1]{\frac{1}{\sqrt{2\pi}#1}e^{-x^2/(2#1^2)}}
\newcommand{\norms}[0]{\frac{1}{\sqrt{2\pi}}e^{-x^2/2}}
\newcommand{\dconverge}[0]{\overset{\mathcal{D}}{\to}}
\newcommand{\pconverge}[0]{\overset{\mathcal{P}}{\to}}
\newcommand{\dd}[0]{\mathrm{d}}
\title{\vspace{-5ex}}
\author{基科32 曾柯又 2013012266}
\date{\vspace{-5ex}}
\begin{document}
\abovedisplayskip=5pt
\belowdisplayskip=5pt
\abovedisplayshortskip=0pt
\belowdisplayshortskip=0pt
\maketitle
\paragraph{5.61}
\subparagraph{(a)}\(\displaystyle
f_Y(y) = \frac{1}{B(a,b)}y^{a - 1}(1 - y)^{b - 1} \quad 0 \leq  y \leq 1\\
f_V(y) = \frac{1}{B([a],[b])}y^{[a] - 1}(1 - y)^{[b] - 1}\\
\frac{f_Y(y)}{f_V(y)} = \frac{B(a,b)}{B([a],[b])}y^{a - [a]}(1 - y)^{b - [b]} \quad 0 \leq  y \leq 1\\
\because a - [a] \in [0,1) ,\quad b - [b] \in [0,1)\\
\therefore y^{a - [a]} ,\; (1 - y)^{b - [b]}\in [0,1]\\
\)
即 \(\displaystyle M = sup\frac{f_Y(y)}{f_V(y)}\) 有界。
\subparagraph{(b)}这小问题目有问题,仅当V的参数\(b'\)大于\(b\),或者\(a'\)等于\(a\)时,M才为有限值。如果将V的参数改为([a],b + 1)则\\
\(\displaystyle f_Y(y) = \frac{1}{\Gamma(a)b^a}y^{a - 1}e^{-\frac{y}{b}}\\
f_V(y) = \frac{1}{\Gamma([a])b^{[a]}}y^{[a] - 1}e^{-\frac{y}{b + 1}}\\
\frac{f_Y(y)}{f_V(y)} = \frac{\Gamma([a])(b + 1)^{[a]}}{\Gamma(a)b^a}y^{a - [a]}e^{-y(\frac{1}{b} - \frac{1}{b + 1})}\)
\subparagraph{(c)}
若将V的参数换成\([a] + 1\), (a),(b)中\(y\)对应的指数\(a - [a] - 1\)变为负,于是当\(y \to 0\) 时,\(\frac{f_Y(y)}{f_V(y)}\to \infty \),即M 会变为无穷大。
\subparagraph{(d)}
如果仅仅是为了使\(EN\)最小,显然当V,Y的分布一样时,M最小为1。但是考虑到问题的实际背景,即为了通过简单的容易产生的随机数来产生复杂的随机数。而由前面的例题可以知道,可以用均匀分布随机变量较容易的产生出两个参数都为整数的Beta分布以及\(\alpha\)参数为整数的Gamma分布。因此在寻求使\(EN\)最小的参数时,应当将范围限制在上述允许的条件内。

对Beta分布,设V的参数为\((a',b')\),则\\
\(\frac{f_Y(y)}{f_V(y)} = \frac{B(a,b)}{B(a',b')}y^{a - a'}(1 - y)^{b - b'}\)\\
记\(\alpha = a - a',\; \beta = b - b'\),可知当\(y = \frac{\alpha}{\alpha + \beta}\)时,有最大值\\
\(M = \frac{B(a,b)}{B(a',b')}(\frac{\alpha}{\alpha + \beta})^{\alpha}(\frac{\beta}{\alpha + \beta})^{\beta}\)\\
通过计算可以知道,\(\alpha , \beta\)越小时,M越小。而为使\(a',\;b'\)都为整数,因此应当取\(a' = [a],\; b' = [b]\)

对Gamma分布,设V的参数为\((a',b')\),则\\
\(
\displaystyle
\frac{f_Y(y)}{f_V(y)} = \frac{\Gamma(a')(b')^{a'}}{\Gamma(a)b^a}y^{a - a'}e^{-y(\frac{1}{b} - \frac{1}{b'})}\)
记\(\alpha = a - a' ,\; \frac{1}{\beta} = \frac{1}{b} - \frac{1}{b'}\),则可以知道当\(y = \alpha\beta\)时,取得最大值\( M = \frac{\Gamma(a')(b')^{a'}}{\Gamma(a)b^a}(\alpha\beta)^{\alpha}e^{-\alpha}\),为求得该式的最小值。对\(a'\)来说,\(a'\)越小,对应的极小值越小,因此
应当取\(a' = [a]\),但是对于\(b'\),无法解析的给出取最小对应的值,只有具体情况数值求解。
\paragraph{5.64}
(a)记\(A = suppf = suppg \),则
\begin{flalign*}
\begin{split}
E\Big(\frac{1}{m}\sum_{i = 1}^{m}\frac{f(Y_i)}{g(Y_i)}h(Y_i)\Big) & = E\frac{f(Y_i)}{g(Y_i)}h(Y_i)\\
& = \int_{A}\frac{f(y)}{g(y)}h(y)g(y)\mathrm{d}y\\
& = \int_{A}f(y)h(y)\mathrm{d}y\\
& = Eh(X)
\end{split}&
\end{flalign*}
(b)记\(\displaystyle T_i = \frac{f(Y_i)}{g(Y_i)}h(Y_i)\)\\
\(
\displaystyle
E|T_i| = \int_{A}|\frac{f(y)}{g(y)}|h(y)g(y)\dd y = E|h(X)|
\)
由大数定律 :
\(\displaystyle \frac{1}{m}\sum_{i = 1}^{m}T_i \overset{p}{\to} ET_i = Eh(X)\)\\
(c)记\(\displaystyle A_i = \frac{f(Y_i)}{g(Y_i)} ,B_i = \frac{f(Y_i)}{g(Y_i)}h(Y_i)\\
EA_i = 1 ,\;E|A_i| = 1 \text{ 由大数定律 } \bar{A}_m \overset{p}{\to} 1\\
EB_i = h(X) ,\; E|B_i| < \infty \text{ 由大数定律 } \bar{B}_m \overset{p}{\to} Eh(X)\)\\
因此\(\displaystyle \frac{\bar{B}_m}{\bar{A}_m} \overset{p}{\to} \frac{Eh(X)}{1} = Eh(X) \),即\(\displaystyle \sum_{i = 1}^{m}\frac{f(Y_i)/g(Y_i)}{\sum_{j = 1}^{m}f(Y_j)/g(Y_j)}h(Y_i) \overset{p}{\to} Eh(X) \)
(c)当h是常数时,\(Eh(X) = h\),(c)给出的估计就为\(h = Eh(X)\),而(b)给出的估计量为\(\displaystyle Eh(X)\frac{1}{m}\sum_{i = 1}^{m}\frac{f(Y_i)}{g(Y_i)}\),因此在这个特殊情况下,(c)给出的估计量比(b)中公式更好。	
\paragraph{模拟作业}
\subparagraph{[1]}
由\(\displaystyle \int_{0}^{2}g(x)\mathrm{d}x = \frac{3}{2}\)故令\(K = \frac{2}{3}\),此时\(f(x) = Kg(x)\)为密度函数
\subparagraph{[2]}
\(0 < x < 0.5\)时
\[F(x) = \int_{0}^{x}\frac{2}{3}2x\dd x = \frac{2x^2}{3} \]
\(0.5 < x < 1\)时
\[F(x) = \frac{1}{6} + \int_{0.5}^{x}\frac{2}{3}(2 - 2x)\dd x = \frac{1}{3} - \frac{2}{3}(x - 1)^2\]
\(1 < x < 1.5\)时
\[F(x) = \frac{1}{3} + \int_{0}^{x - 1}\frac{2}{3}4x\dd x = \frac{1}{3} + \frac{4}{3}(x - 1)^2\]
\(1.5 < x < 2\)时
\[F(x) = \frac{2}{3} + \int_{1.5}^{x}\frac{2}{3}(8 - 4x)\dd x = 1 - \frac{4}{3}(x - 2)^2  \]
做出\(f(x),F(x)\)的图形如下:\\
\includegraphics[scale=0.5]{Rf}\includegraphics[scale=0.5]{RF}\\
计算\(\displaystyle \mu = EX = \frac{7}{6}\) , \(\displaystyle\sigma^2 = VarX = \frac{19}{72} \)
\subparagraph{[3]}
利用直接方法产生随机数:\\
\indent 用分段函数表示\(F^{-1}(x)\),作用于产生的1000个(0,1)均匀分布随即变量,得到的1000个数即满足\(f(x)\)的分布,计算所得的\(\hat{\mu}_1 = 1.179\)
\subparagraph{[4]}
利用接收-拒绝法产生随机数:\\
\indent 这里我为了避免使用循环语句,而尽可能向量化,采取了先生成两列足够多的均匀分布随机数列,用向量操作筛选出满足\(y < f(x)\)数据的方法,但是由于向量长度是固定的,还要提前取定足够的数据使得筛选后至少有1000个数。这可以估计如下,满足要求的数分布在\(f(x)\)下方,面积为1,而所有数据在大方框内,满足要求概率为\(\frac{3}{8}\),要得到1000个数据,可以先生成2800组随即数,即基本能保证数据量,这样做虽然会额外生成一些数,但是在循环次数更多的时候更能节省时间。\\
两种方法获得直方图与概率密度函数叠加所得图如下\\
{\centering \includegraphics[scale = 0.5]{Rdirect} \includegraphics[scale=0.5]{Raccept}}
\subparagraph{[5]}
两种方法各重复2000次:
由于每次重复的步骤都是一样的,我没有写成循环,而是直接产生\(1000\times2000\)组数据,再分成2000组,分组求平均,这样与循环2000次是一样的,速度也要快一点。
由大数定律,这2000个值近似服从正态分布,即\(\displaystyle \frac{\sqrt{1000}(\hat{\mu} - \mu)}{\sigma} \sim n(0,1)\),可以做图来验证这一点,即这两千个数的直方图应当和正态分布密度函数相合。\\
两种方法得到的图像如下\\
{\centering \includegraphics[scale = 0.5]{norm1} \includegraphics[scale=0.5]{norm2}}\\
两种方法得到的样本均值与方差为:\\
Direct method : \(\bar{\hat{\mu_1}}=1.16631 ,\;S^2_1 = 2.57\times10^{-4}\)\\
Accept Reject : \(\bar{\hat{\mu_2}}=1.16673 ,\;S^2_2 = 2.56\times10^{-4}\)\\
\subparagraph{[6]}
\indent 记时两种算法运行时间:都产生2000组平均值,直接方法用时30.401s,接受拒绝法用时75.579s,因此直接方法明显节约了很多时间。但这和问题的特殊性有关,因为cdf的反函数可以显示表达,在大多数情况下,cdf并不能直接表达,而获取反函数需要解方程,而且这个方程一般会涉及积分,因此方程的求解会异常耗时。所以在反函数不能直接表达的时候,接受拒绝法要可行。
\end{document}
