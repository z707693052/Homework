\documentclass[11pt,a4paper]{ctexart}
\usepackage{amsmath,amssymb,amsthm}
\usepackage{graphicx}
\DeclareGraphicsExtensions{.pdf,.png,.jpg,.eps,.svg}
\linespread{1.5}
\CTEXsetup[indent={0pt}]{subparagraph}
\newcommand{\normal}[2]{\frac{1}{\sqrt{2\pi}#2}e^{-(x - #1)^2/(2#2^2)}}
\newcommand{\norm}[1]{\frac{1}{\sqrt{2\pi}#1}e^{-x^2/(2#1^2)}}
\newcommand{\norms}[0]{\frac{1}{\sqrt{2\pi}}e^{-x^2/2}}
\newcommand{\dconverge}[0]{\overset{\mathcal{D}}{\to}}
\newcommand{\pconverge}[0]{\overset{\mathcal{P}}{\to}}
\newcommand{\dd}[0]{\mathrm{d}}
\title{\vspace{-5ex}}
\author{基科32 曾柯又 2013012266}
\date{\vspace{-5ex}}
\begin{document}
	\graphicspath{{C:/Users/cengqQ/Pictures/}}
\abovedisplayskip=5pt
\belowdisplayskip=5pt
\abovedisplayshortskip=0pt
\belowdisplayshortskip=0pt
\maketitle
\paragraph{1}
R中的pnorm函数直接给出了cdf,可以计算得\(\mathrm{C}\approx 1.0001397\),\(f(x)\)的函数图像如下\\
{\centering \includegraphics[scale = 0.5]{fplot}}
\paragraph{2}
画出图像:
{\center \includegraphics[scale = 0.4]{f100}\includegraphics[scale = 0.4]{f1000}}
可以看出抽样结果是正确的
\end{document}
