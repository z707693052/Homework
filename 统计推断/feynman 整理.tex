\documentclass[10pt,a4paper,nocap]{ctexart}
\usepackage{amsmath,amssymb,amsthm}
\usepackage{textcomp}
\usepackage{graphicx}
\usepackage{hyperref}
\CTEXsetup[titleformat={\large}]{section}
\title{Feynman 11月4日整理}
\author{整理人 :  曾柯又}
\date{\vspace{-5ex}}
\linespread{1.5}
\newcommand{\ket}[1]{\left|#1\right>}
\newcommand{\bra}[1]{\left<#1\right|}
\newcommand{\braket}[2]{\left<#1|#2\right>}
\begin{document}
\maketitle
\section{两粒子态}
\subsection{数学表示}
最简单的两粒子态可以写为直积(张量积)的形式:\[\left|a\right>_1\left|b\right>_2 = \left|a\right>\left|b\right> = \left|a\right>\otimes\left|b\right> \]
解释为:粒子1 处于态\(\left|a\right>\)上且粒子2处于态\(\left|b\right>\)上的态。

两粒子可以构成的基础态:\(\left|+\right>\left|+\right>,\left|+\right>\left|-\right>,\left|-\right>\left|+\right>,\left|-\right>\left|-\right>\)。
由线性叠加原理,这些基础态的任意线性组合都是物理上允许的态。例如\(\frac{1}{\sqrt{2}}(\left|+\right>\left|+\right> + \left|-\right>\left|-\right>)\)但是注意此时不能再称粒子一处于某态,粒子二处于某态。
\subsection{内积定义}
记\(\left|\phi_a\right> = \left|a_1\right>\left|a_2\right>,\;\left|\phi_b\right> = \ket{b_1}\ket{b_2}\),则\(\braket{\phi_a}{\phi_b} = \braket{a_1}{b_1}\braket{a_2}{b_2}\)。
这个内积依然解释为,对于\(\left|\phi_b\right>\)的两粒子态,测得\(\left|\phi_a\right>\)的概率幅。
\section{纠缠态}
\subsection{纠缠态的定义}
在仅限与纯态的情况,当两粒子的态能够写成下列形式:\(\ket{\psi} = \ket{a}\ket{b}\),则为非纠缠的,若不能则为纠缠态。
\subsection{最大纠缠态}
数学表示\[\ket{\phi^\pm} = \frac{1}{\sqrt{2}}(\ket{++} \pm \ket{--})\]
\[\ket{\psi^\pm} = \frac{1}{\sqrt{2}}(\ket{+-} \pm \ket{-+})\]
由前面定义的内积,这四个态是正交归一的,因此可以作基础态和测量基。\\
\subsection{纠缠态与通信}
纠缠态的特有属性:关联。以态\(\ket{\phi^+} = \frac{1}{\sqrt{2}}(\ket{++} + \ket{--})\)为例:A,B两个粒子可相隔很远。若一端(A端)得\(\ket{+}\),则另一端也为\(\ket{+}\),若A端\(\ket{-}\),B则端也是
\(\ket{-}\)

注意:纠缠态并不能实现超光速通信。

解释:首先回答什么是通信,如果你给别人打电话,不管你说什么,别人始终在放我的滑板鞋,这显然不是通信。。。囧Orz。
而对纠缠态系统\(\ket{\phi^+}\)来说,1粒子没有测量时,对于粒子2而言,其密度矩阵为\(I/2\)。 而对粒子一,无论做何种测量,只要没有把测量结果传到粒子2端,则粒子2的密度矩阵仍然为\(I/2\)。

也就是说在没有经典通信的情况下,1粒子端的任何测量都不会对2粒子端的测量产生影响。也就相当于洗脑循环我的滑板鞋了,这当然就不会有任何信息的传递。更多的关于纠缠态的讨论见ppt7。
\section{Bell测量}
Bell测量就是以Bell态为测量基的测量,它是非局域测量或者叫集体测量,不能分解成单个子测量的组合。
\subsection{CNOT门}
可以利用CNOT门(control not gate)来实现bell测量。CNOT门的定义为: ,对两粒子体系\(\ket{a}\ket{b}\):\[\mathrm{CNOT}\ket{a}\ket{b} = \ket{a}\ket{a\oplus b}\]
\(\oplus\)为逻辑异或操作。可以利用CNOT门来实现bell态的测量。
CNOT门对bell态的作(用+表示1,-表示0)用:
\begin{equation}
\begin{split}
&\mathrm{CNOT}\ket{\phi^+} = (\frac{1}{\sqrt{2}}(\ket{+}+\ket{-}))\ket{-}\\
&\mathrm{CNOT}\ket{\phi^-} =  (\frac{1}{\sqrt{2}}(\ket{+}-\ket{-}))\ket{-}\\
&\mathrm{CNOT}\ket{\psi^+} =  (\frac{1}{\sqrt{2}}(\ket{+}+\ket{-}))\ket{+}\\
& \mathrm{CNOT}\ket{\psi^-} = (\frac{1}{\sqrt{2}}(\ket{+}-\ket{-}))\ket{+}\\
\end{split}
\end{equation}
通过CNOT门之后,将纠缠态转变为了非纠缠态,将Bell基的测量变成了对水平方向和45\textdegree 度以及135\textdegree 的测量。
\section{纠察态的应用}
\subsection{Quantum Teleportation}
可以将一个未知态传输到远方而不传输粒子本身。

一个3粒子态,若能写成:\[\ket{\chi}_{123} = c_1\ket{\psi_1}_{12}\ket{a}_3 + c_2\ket{\psi_2}_{12}\ket{b}_3 + c_3\ket{\psi_3}_{12}\ket{c}_3 + c_4\ket{\psi_4}_{12}\ket{d}_3\]
且\(\ket{\psi_i}\)正交归一,则以测量基\(\ket{\psi_i}\)观测12,那么测量结果决定了粒子3的状态。

该结论要用到Schmidt decomposition,可以见wiki百科\url{http://en.wikipedia.org/wiki/Schmidt_decomposition}

Quantum Teleportation的具体数学推导见ppt7

注意Quantum Teleportation仍然不能用来做超光速通信。
\subsection{量子密集编码}
可用1个粒子来传递2个bit的信息。数学推导见ppt7

鲍亦澄同学的提问:实际上B粒子也要送过去,仍然是用2个粒子传递2bit的信息。

老师的解释:的确实际上是用2粒子传递2bit信息,但是B粒子可以提前传送,可以利用信道闲置的时间提前发送粒子B,这样就可以在通信繁忙的时候利用1个粒子加载2bit的信息。
\end{document}
