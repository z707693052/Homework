\documentclass[11pt,a4paper]{ctexart}
\usepackage{amsmath,amssymb,amsthm}
\usepackage{graphicx}
\usepackage{environ}
\DeclareGraphicsExtensions{.pdf,.png,.jpg,.eps}
\graphicspath{{/home/cengq/Pictures/}}
\linespread{1.6}
\CTEXsetup[indent={0pt}]{subparagraph}
\newcommand{\normal}[2]{\frac{1}{\sqrt{2\pi}#2}e^{-(x - #1)^2/(2#2^2)}}
\newcommand{\norm}[1]{\frac{1}{\sqrt{2\pi}#1}e^{-x^2/(2#1^2)}}
\newcommand{\norms}[0]{\frac{1}{\sqrt{2\pi}}e^{-x^2/2}}
\newcommand{\dconverge}[0]{\overset{\mathcal{D}}{\to}}
\newcommand{\pconverge}[0]{\overset{\mathcal{P}}{\to}}
\newcommand{\dd}[0]{\mathrm{d}}
\NewEnviron{formula}{ 
\begin{flalign*} \begin{split}
\BODY
\end{split}&\end{flalign*}}
\title{\vspace{-5ex}}
\author{基科32 曾柯又 2013012266}
\date{\vspace{-5ex}}
\begin{document}
\abovedisplayskip=5pt
\belowdisplayskip=5pt
\abovedisplayshortskip=0pt
\belowdisplayshortskip=0pt
\maketitle
\paragraph{7.53}
考虑 \(\phi_a = W + aU\)
\begin{formula}
\text{则}E_\theta\phi_a &= E_\theta W + aE_\theta U\\
 & = \tau(\theta)
\end{formula}
故\(\forall\; a,\;\phi_a\)为\(\tau(\theta)\)的无偏估计量。
\begin{formula}
\text{而}Var_\theta\phi_a & = Var(W + aU)\\
 &= Var_\theta W + aCov_\theta(W,U) + a^2Var_\theta U
\end{formula}
如果存在\(\theta_0\)使得\(Cov_{\theta_0}(W,U) \neq 0\)则可以取\(a\)使得\(aCov_{\theta_0}(W,U) + a^2Var_{\theta_0}(U) < 0\)\\
即若\(Cov_{\theta_0}(W,U) > 0\)\\
取\(\displaystyle a \in (-\frac{Cov_{\theta_0}(W,U)}{Var_{\theta_0}W},0)\)\\
若\(Cov_{\theta_0}(W,U) < 0\)\\
取\(\displaystyle a \in (0,-\frac{Cov_{\theta_0}(W,U)}{Var_{\theta_0}W})\)\\
此时\(Var_{\theta_0}\phi_a < Var_{\theta_0}W\),即\(W\) 不为最佳无偏估计
\paragraph{7.57}
\subparagraph{(a)}
\(ET = \times P(\sum_{i = 1}^{n}X_i > X_{n + 1}|p) = h(p)\)
\subparagraph{(b)}
\(X_1,\dots X_{n + 1}\)的联合分布为:\\
\(\displaystyle f(x_1,\cdots,x_{n + 1}) = p^{\sum_{i = 1}^{n + 1}x_i}(1 - p)^{n + 1 - \sum_{i = 1}^{n+1}x_i}\)

由因子分解定理\(\sum_{i = 1}^{n+1}X_i\)为一个充分统计量,且由于伯努力分布为指数分布族且参数空间包含了\(\mathbb{R}\)的开集,故\(T = \sum_{i = 1}^{n+1}X_i\)为完全充分统计量,因此\(\phi(\sum_{i = 1}^{n + 1}X_i) = E(T|\sum_{i = 1}^{n + 1}X_i)\)为最佳无偏估计。

记\(\displaystyle T_2 = \sum_{i = 1}^{n + 1}X_i\\
P(\sum_{i = 1}^{n}X_i > X_{n + 1} | T_2 = t_2) =\frac{ P(\sum_{i = 1}^{n}X_i > X_{n + 1},T_2 = t_2)}{P(T_2 = t_2)}\\
T_2 = 0\text{时,显然 } P(\sum_{i = 1}^{n}X_i > X_{n + 1},T_2 = 0) = 0 \\
\Rightarrow P(\sum_{i = 1}^{n}X_i > X_{n + 1} | T_2 = 0) = 0\\
T_2 = 1\text{时:   } P(\sum_{i = 1}^{n}X_i > X_{n + 1},T_2 = 1) = (1 - p){n \choose 1}p(1 - p)^{n - 1}\\
\text{而 } P(T_2 = 1) = {n+1 \choose 1}p(1 - p)^n \\
\Rightarrow P(\sum_{i = 1}^{n}X_i > X_{n + 1} | T_2 = 1) = \frac{n}{n+1} \\
T_2 = 2 \text{时} P(\sum_{i = 1}^{n}X_i > X_{n + 1},T_2 = 2) = (1 - p){n \choose 2}p^2(1 - p)^{n - 2}\\
\text{而} P(T_2 = 2) =  {n+1 \choose 2}p^2(1 - p)^{n - 1}\\
\Rightarrow   P(\sum_{i = 1}^{n}X_i > X_{n + 1} | T_2 = 2) = \frac{n - 1}{n + 1}\\
T_2 > 2\text{时,容易看出}P(\sum_{i = 1}^{n}X_i > X_{n + 1} | T_2 ) \equiv 1
\)\\
因此\(\displaystyle \phi(T_2) = \begin{cases}
0 & T_2 = 0\\
\frac{n}{n+1}& T_2 = 1\\
\frac{n - 1}{n +1} & T_2 = 2\\
1 & T_2 > 2
\end{cases}\)
为最佳无偏估计
\paragraph{7.58}
\subparagraph{(a)}
\(L(\theta|x) = (\frac{\theta}{2})^{|x|}(1 - \theta)^{1 - |x|}\\
\log L = |x|\log(\frac{\theta}{2}) + (1 - |x|)\log (1 - \theta)\\
\text{由} \frac{\partial}{\partial \theta}(\log L) = \frac{|x|}{\theta} - \frac{1 - |x|}{1 - \theta} = 0 \\
\Rightarrow \theta = |x|\\
\text{且}\frac{|x|}{\theta} - \frac{1 - |x|}{1 - \theta}\text{单调递减,故极值为极大值}\\
\text{因此 } \hat{\theta} = |x| \text{为极大似然估计}
\)
\subparagraph{(b)}
\(ET(X) = 2P(X = 1) = \theta\)\
\subparagraph{(c)}
由因子分解,\(|x|\)是\(\theta\)的充分统计量,考虑\(\phi(|x|) = E(T\big||x|) \)\\
\(\phi(0) = E(T|x = 0) = 0\\
\phi(1) = E(T\big||x| = 1) = 2P(x = 1\big||x| = 1) = 2\frac{P(x = 1)}{|x| = 1} = 2\frac{(\frac{\theta}{2})}{2\frac{\theta}{2}} = 1
\)
即\(\phi(|x|) = |x|\)是一个更好的\(\theta\)的无偏估计量
\end{document}