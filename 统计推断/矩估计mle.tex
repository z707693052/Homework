\documentclass[11pt,a4paper]{ctexart}
\usepackage{amsmath,amssymb,amsthm}
\linespread{1.9}
\CTEXsetup[indent={0pt}]{subparagraph}
\newcommand{\normal}[2]{\frac{1}{\sqrt{2\pi}#2}e^{-(x - #1)^2/(2#2^2)}}
\newcommand{\norm}[1]{\frac{1}{\sqrt{2\pi}#1}e^{-x^2/(2#1^2)}}
\newcommand{\norms}[0]{\frac{1}{\sqrt{2\pi}}e^{-x^2/2}}
\newcommand{\dconverge}[0]{\overset{\mathcal{D}}{\to}}
\newcommand{\pconverge}[0]{\overset{\mathcal{P}}{\to}}
\newcommand{\bx}[0]{\mathbf{x}}
\title{\vspace{-5ex}}
\author{基科32 曾柯又 2013012266}
\date{\vspace{-5ex}}
\begin{document}
\abovedisplayskip=5pt
\belowdisplayskip=5pt
\abovedisplayshortskip=0pt
\belowdisplayshortskip=0pt
\maketitle
\paragraph{7.1}
当 \(x = 0 \quad \hat{\theta} = 1\\
x = 1 \quad \hat{\theta} = 1\\
x = 2 \quad \hat{\theta} = 2 \text{或} 3 \\
x = 3 \quad \hat{\theta} = 3\\
x  = 4 \quad \hat{\theta} = 3
\)
\paragraph{7.3}
因为\(\log x\)是严格单调的,且由\(L(\theta|\mathbf{x})\)的定义可知其大于零,因此\[L(\theta_1|\mathbf{x}) > L(\theta_2|\mathbf{x}) \Leftrightarrow \log L(\theta_1| \mathbf{x}) > \log L(\theta_2 | \mathbf{x})\]
即\[\forall \theta \neq \theta_0 \quad L(\theta|\mathbf{x}) < L(\theta_0|\mathbf{x}) \Leftrightarrow \forall \theta \neq \theta_0 \quad \log L(\theta|\mathbf{x}) < \log L(\theta_0|\mathbf{x}) \]
\paragraph{7.10}
\subparagraph{(a)} \(\displaystyle f(x|\alpha,\beta) = \alpha\frac{x^{\alpha}}{\beta^{\alpha}}I(0 \leq x \leq \beta)\)\\
因此\begin{flalign*}
\begin{split}
f(\mathbf{x}|\alpha,\beta) & = (\frac{\alpha}{\beta})^n(\prod_{i = 1}^{n}x_i)^{\alpha - 1}\prod_{i = 1}^{n}I(0 \leq x_i \leq \beta)\\
& = (\frac{\alpha}{\beta})^n(\prod_{i = 1}^{n}x_i)^{\alpha - 1}I(x_{(n)} < \beta)I(x_{(1)} > 0 )
\end{split}
\end{flalign*}
记\(\displaystyle t_1 = \prod_{i = 1}^{n}x_i\quad t_2 = x_{(n)}\quad f(\mathbf{x}|\alpha,\beta) = (\frac{\alpha}{\beta})^nt_1^{\alpha - 1}I(t_2 < \beta)I(x_{(1)} > 0 )\)\\
由因子分解定理\(T(\mathbf{X}) = (t_1,t_2)\),为一个充分统计量。
\subparagraph{(b)}
 记函数\(g(\alpha,\beta) = (\frac{\alpha}{\beta^{\alpha}})^nt_1^{\alpha - 1}I(t_2 \leq \beta)\)
 首先对任意\(\alpha\),易看出\(\beta\)越小,\(g\)越大,而\(\beta\)小于\(t_2\)时变为0,因此\(\hat{\beta} = t_2\)\\
 而对\(\displaystyle g(\alpha,\hat{\beta})= \frac{1}{\hat{\beta}^n}\alpha^n(\frac{t_1}{\hat{\beta}^n})^{\alpha - 1}\)\\
 \(\displaystyle
 \frac{\partial}{\partial \alpha}\log g(\alpha,\hat{\beta}) = \frac{n}{\alpha} + \log (\frac{t_1}{\hat{\beta}}^n)\\
 \because t_1 < t_2^n = \hat{\beta}^n \; \Rightarrow \; \frac{t_1}{\hat{\beta}^n} < 1 \\
 \text{且} \frac{n}{\alpha} + \log (\frac{t_1}{\hat{\beta}})^n \text{随} \alpha \text{单减,故对应极值为极大值 } \\
 \therefore \hat{\alpha} = \frac{n}{\log (\frac{\hat{\beta}^n}{t_1})} = \frac{n}{\log (\frac{t_2^n}{t_1})}\)
\subparagraph{(c)} \(\displaystyle n = 14\quad x_{(n)} = 25 \text{ 即 } \hat{\beta} = 25\\
 \hat{\alpha} = \frac{n}{n\log x_{(n)} - \log t_1} = \frac{1}{\log x_{(n)} - \frac{1}{n}\sum_{i = 1}^{n}\log x_i} \approx 12.595
\)
\paragraph{7.11}
\subparagraph{(a)}
\(L(\theta|\mathbf{x}) = \theta^n(\prod_{i = 1}^{n}x_i)^{\theta - 1} \)
记 \(\displaystyle t = \prod_{i = 1}^{n}x_i \\
 L(\theta|\mathbf{x}) = \theta^nt^{\theta - 1}\\
\frac{\partial \log L}{\partial \theta} = \frac{n}{\theta} + \log t\\
\because t < 1 \text{ 且 } \frac{n}{\theta} + \log t \text{单调递减, 故极值为极大值,因此 }\hat{\theta} = \frac{n}{\log (\frac{1}{t})} \)\\
容易证明 
\(\displaystyle - \log X_i \sim Exp(\frac{1}{\theta}) \quad \text{故 } -\sum_{i = 1}^{n}\log X_i \sim Gamma(n,\frac{1}{\theta})\\
\text{因此 }\frac{\hat{\theta}}{n} \sim \frac{1}{x^2}\frac{\theta^n}{(n - 1)!}(\frac{1}{x})^{n - 1}e^{- \frac{\theta}{x}}\\
E\frac{\hat{\theta}}{n} = \frac{\theta}{n - 1} \\
 Var \frac{\hat{\theta}}{n} = E\Big(\frac{\hat{\theta}}{n}\Big)^2 - \Big(E\frac{\hat{\theta}}{n}\Big)^2 = \frac{\theta^2}{(n - 1)(n - 2)} - \frac{\theta^2}{(n - 1)^2} = \frac{\theta^2}{(n - 1)^2(n - 2)}\\
 \text{可得} Var \hat{\theta} = \frac{n^2\theta^2}{(n - 1)^2(n - 2)} \Rightarrow \; n \to \infty \text{ 时 }   Var \hat{\theta} \to 0 \)
 \subparagraph{(b)}
 \(\displaystyle EX_i = \frac{\theta}{\theta + 1} \Rightarrow \frac{1}{n}\sum_{i = 1}^{n}X_i = \frac{\tilde{\theta}}{\tilde{\theta} + 1} \\ 
 \tilde{\theta} = \frac{\bar{x}}{1 - \bar{x}}\)
 因为\(\displaystyle X_i \leq 1 \; \Rightarrow \bar{x} \leq 1 \\
 \text{又 } P(\bar{x} = 1) = 0, 因此 \tilde{\theta} = \frac{\bar{x}}{1 - \bar{x}} > 0\)
\paragraph{7.12}
\subparagraph{(a)} \(EX = \theta\)
用矩估计的方法可得
\(\tilde{\theta} = \frac{1}{n}\sum_{i = 1}^{n}X_i\)\\
用MLE估计:\\
\(L(\theta|\bx) = \theta^{\sum_{i = 1}^{n}x_i}(1 - \theta)^{n - \sum_{i = 1}^{n}x_i}\)\\
记\(t = \frac{1}{n}\sum_{i = 1}^{n}x_i\)\\
则\(L(\theta|\bx) = (\theta^t(1 - \theta)^{1 - t})^n\)\\
\(\displaystyle \Rightarrow \frac{1}{n}\frac{\partial L(\theta|\bx)}{\partial \theta} = \frac{t}{\theta} - \frac{1 - t}{1 - \theta}\)\\
该式随\(\theta\)单减,于是\(L(\theta|\bx)\)在\(\theta = t\) 时取得极大值。但是考虑到\(\theta < \frac{1 }{2}\),故当\(t > \frac{1}{2} \)时\(L(\theta|\bx)\)在区间\([0,\frac{1}{2}]\)上单调递增,\(\theta\)只能取\(\frac{1}{2}\),\(L(\theta|\bx)\)最大,因此\\
\(\hat{\theta} = \begin{cases}
\frac{1}{2} & t > \frac{1}{2}\\
t & t \leq \frac{1}{2}
\end{cases}\)
\subparagraph{(b)}
对于矩估计\(\tilde{\theta} = \frac{1}{n}\sum_{i = 1}^{n}X_i\),而\(\sum_{i = 1}^{n}X_i \sim B(n,\theta)\),因此\\
\(E\tilde{\theta} = \frac{1}{n}\times n\theta = \theta\)\\
\(\displaystyle MSE(\tilde{\theta}) = Var\tilde{\theta} = \frac{1}{n^2}Var\sum_{i = 1}^{n}X_i = \frac{\theta(1 - \theta)}{n}\)\\
对于MLE估计:
\begin{flalign*}
\begin{split}
MSE(\hat{\theta})& = E(\hat{\theta} - \theta)^2 \\
& = \sum_{k = 1}^{n}(\hat{\theta} - \theta)^2{n \choose k}\theta^k(1 - \theta)^{1 - k}\\
& = \sum_{k = 1}^{[\frac{n}{2}]}(\frac{k}{n} - \theta)^2{n \choose k}\theta^k(1 - \theta)^{n - k}
+ \sum_{k = [\frac{n}{2}] + 1}^{n}(\frac{1}{2} - \theta)^2{n \choose k}\theta^k(1 - \theta)^{n - k}
\end{split}&
\end{flalign*}
\subparagraph{(c)}
\(MSE(\tilde{\theta}) = \sum_{k = 1}^{n}(\frac{k}{n} - \theta)^2{n \choose k}\theta^k(1 - \theta)^{n - k}
\)
因此
\begin{flalign*}
\begin{split}
MSE(\tilde{\theta})& - MSE(\hat{\theta}) = \\
& \sum_{k = [\frac{n}{2}] + 1}^{n}\big((\frac{k}{n} - \theta)^2 - (\frac{1}{2} - \theta)^2\big){n \choose k}\theta^k(1 - \theta)^{n - k}\\
& = \sum_{k = [\frac{n}{2}] + 1}^{n}(\frac{k}{n} + \frac{1}{2} - 2\theta)(\frac{k}{n} - \frac{1}{2}){n \choose k}\theta^k(1 - \theta)^{n - k}
\end{split}
\end{flalign*}
\(\displaystyle \because \theta < \frac{1}{2} , \; \frac{k}{n} > \frac{1}{2} \\
 \therefore MSE(\tilde{\theta}) - MSE(\hat{\theta}) > 0\)\\
 即最似然估计更好
\paragraph{7.13}
\(L(\theta|\bx) = \frac{1}{2}e^{-\sum_{i = 1}^{n}|x_i - \theta|}\)\\
对于\(L' = \sum_{i = 1}^{n}|x_i - \theta| = \sum_{i = 1}^{n}|x_{(i)} - \theta|\)\\
易知\(|x_{(i)} - \theta| + |x_{(j)} - \theta| \geq |x_{(i)} - x_{(j)}|\),仅当\(\theta \text{在} x_{(i)}, \; x_{(j)}\)之间等号成立\\
因此当\(n\)为奇数时,容易知道当\(\displaystyle \hat{\theta} = m = x_{(\frac{n + 1}{2})}\)时,\(L'\)取得最小,对应\(L\)取得最大
当\(n\)为奇数时,有\(\displaystyle \hat{\theta} \in (x_{(\frac{n }{2})}, x_{(\frac{n}{2} + 1)}) \)时,\(L'\)取得最小,对应\(L\)取得极大值。
\paragraph{7.14}
\begin{flalign*}
\begin{split}
P(Z < z,W = 1) & = P(X \leq z,X \leq Y)\\
& = \int_{0}^{z}\mathrm{d}x\int_{x}^{\infty}\mathrm{d}y \frac{1}{\lambda\mu}e^{-(\frac{x}{\lambda} + \frac{y}{\mu
})}\\
& = \int_{0}^{z}\mathrm{d}x \frac{1}{\lambda}e^{-\frac{x}{\lambda}-\frac{x}{\mu}}\\
& = \frac{1}{1 + \frac{\lambda}{\mu}}(1 - e^{-(\frac{1}{\lambda}+\frac{1}{\mu})z})
\end{split}&
\end{flalign*}
同理\(\displaystyle P(Z < z , W = 0) = \frac{1}{1 + \frac{\mu}{\lambda}}(1 - e^{-(\frac{1}{\lambda}+\frac{1}{\mu})z})\\
f_{Z,W}(z,1) = \frac{\partial}{\partial z}P(Z < z,W = 1) = \frac{1}{\lambda}e^{-(\frac{1}{\lambda}+\frac{1}{\mu})z}\\
f_{Z,W}(z,0) = \frac{\partial}{\partial z}P(Z < z,W = 0) =\frac{1}{\mu}e^{-(\frac{1}{\lambda}+\frac{1}{\mu})z}\\
f_{Z,W}(z,w) = (\frac{1}{\lambda}e^{-(\frac{1}{\lambda}+\frac{1}{\mu})z})^w(\frac{1}{\mu}e^{-(\frac{1}{\lambda}+\frac{1}{\mu})z})^{1 - w} = (\frac{1}{\lambda})^w(\frac{1}{\mu})^{1 - w}e^{-(\frac{1}{\lambda}+\frac{1}{\mu})z}
\Rightarrow\\
L(\lambda,\mu|(\mathbf{z},\mathbf{w})) = (\frac{1}{\lambda})^{\sum_{i = 1}^{n}w_i}(\frac{1}{\mu})^{n - \sum_{i = 1}^{n}w_i}e^{-(\frac{1}{\lambda}+\frac{1}{\mu})\sum_{i = 1}^{n}z_i}\)\\
记 \(t_1 = \sum_{i = 1}^{n}w_i ,\; t_2 = \sum_{i = 1}^{n}z_i,\;\theta_1 = \frac{1}{\lambda} ,\;\theta_2 = \frac{1}{\mu}\)\\
则\(\displaystyle L(\lambda,\mu|(\mathbf{z},\mathbf{w})) = (\theta_1)^{t_1}(\theta_2)^{n - t_1}e^{-(\theta_1 + \theta_2)t_2}\\
\log L = t_1\log\theta_1 + (n - t_1)\log \theta_2 -(\theta_1 + \theta_2)t_2\\
\frac{\partial\log L}{\partial \theta_1} = \frac{t_1}{\theta_1} - t_2\\
\frac{\partial\log L}{\partial \theta_2} = \frac{n - t_1}{\theta_2} - t_2
\)
且两个偏倒数都随\(\theta_1 , \; \theta_2\)单点递减,因此对应极值为极大值,即:\\
\(\displaystyle \theta_1 = \frac{t_1}{t_2} \; \theta_2 = \frac{n - t_1}{t_2}\\
\Rightarrow \hat{\lambda} = \frac{\sum_{i = 1}^{n}z_i}{\sum_{i = 1}^{n}w_i}\quad \hat{\mu} = \frac{\sum_{i = 1}^{n}z_i}{n - \sum_{i = 1}^{n}w_i}\)
\paragraph{7.18}
\subparagraph{(a)}
为简化记号,设\(\mathbf{Z} = (X,Y) ,\mathbf{z} = (x,y) ,\mathbf{\mu_Z} = (\mu_X , \mu_Y)\)\\
 则\(x,y\)的分布为 \[f_\mathbf{Z}(\mathbf{z}) = \frac{1}{2\pi\sqrt{|\Sigma|}}e^{-\frac{1}{2}((\mathbf{z - \mu_Z})\Sigma^{-1}\mathbf{(z - \mu_Z)}^{T})}\]
 其中协方差矩阵为:\(\begin{bmatrix}
 \sigma_X^2 & \rho\sigma_X\sigma_Y\\
 \rho\sigma_X\sigma_Y & \sigma_Y^2
 \end{bmatrix}\)\\
 对\(y\)积分可以得到\(x\)的分布 \(\displaystyle f_X(x) = \frac{1}{\sqrt{2\pi}\sigma_X}e^{-\frac{(x - \mu_X)^2}{\sigma_X^2}}\)\\
 于是 \(EX = \mu_X ,\;EX^2 = VarX + (EX)^2 = \sigma_X^2 + \mu_X^2\\
 \Rightarrow \tilde{\mu}_X = \bar{x} ,\; \tilde{\mu}_X + \tilde{\sigma}_X^2 = \bar{x^2}\\
 \Rightarrow \tilde{\sigma}_X^2 = \bar{x^2} - \bar{x}^2 = \frac{1}{n}\sum_{i = 1}^{n}(x_i - \bar{x})^2\)\\
 同理对于\(y\),有:\\
 \(\tilde{\mu}_Y = \bar{y},\; \tilde{\sigma}_Y^2= \frac{1}{n}\sum_{i = 1}^{n}(y_i - \bar{y})^2\)\\
 而\(\displaystyle \rho = \frac{Cov(X,Y)}{\sqrt{VarX\, Var Y}} = \frac{EXY - EXEY}{\sqrt{VarX\,VarY}}\),得到:\\
 \(\displaystyle EXY = \sigma_X\sigma_Y\rho + \mu_X\mu_Y\\
 \Rightarrow \tilde{\rho}\tilde{\sigma}_X\tilde{\sigma}_Y + \tilde{\mu}_X\tilde{\mu}_Y = \frac{1}{n}\sum_{i = 1}^{n}x_iy_i = \frac{1}{n}\sum_{i = 1}^{n}(x - \bar{x})(y - \bar{y}) + \bar{x}\bar{y}\\
 \Rightarrow \tilde{\rho} = \frac{\frac{1}{n}\sum_{i = 1}^{n}(x_i - \bar{x})(y_i - \bar{y})}{\tilde{\sigma}_X\tilde{\sigma}_Y}\)
 \subparagraph{(b)}
 不采用书中的办法,由于\(\Sigma\)是半正定对称矩阵,于是存在一个正交矩阵\(A\)可将其对角化,即\(\displaystyle A\Sigma A^T = \begin{pmatrix}
 \sigma^2_u & 0\\
 0& \sigma^2_v
 \end{pmatrix}\)\\
 记\(\displaystyle \mathbf{w} = (u, v) = \mathbf{z}A^T , \;\mathbf{\mu_w} = (\mu_u,\mu_v)= \mathbf{\mu_Z}A^T\)
 于是pdf可写为 \begin{flalign*}
 \begin{split}
 f_\mathbf{Z}(\mathbf{z}) & = \frac{1}{2\pi\sqrt{|\Sigma|}}e^{-\frac{1}{2}((\mathbf{w - \mu_w})A\Sigma^{-1}A^T\mathbf{(w - \mu_w)}^{T})} \\
 &= \frac{1}{2\pi\sqrt{|\Sigma|}}e^{-\frac{1}{2}((\mathbf{w - \mu_w})(A\Sigma A^T)^{-1}\mathbf{(w - \mu_w)}^{T})}\\
 & = \frac{1}{2\pi\sqrt{|\Sigma|}}e^{-\frac{1}{2}(\frac{(u - \mu_u)^2}{\sigma_u^2} + \frac{(v - \mu_v)^2}{\sigma_v^2})}\\
 \end{split}&
 \end{flalign*}
 一系列样本值\((x_i,y_i)\),对应着一系列的\((u_i,v_i)\),似然函数:
 \begin{flalign*}
 \begin{split}
 L & = \prod_{i = 1}^{n}\frac{1}{2\pi\sqrt{|\Sigma|}}e^{-\frac{1}{2}(\frac{(u_i - \mu_u)^2}{\sigma_u^2} + \frac{(v_i - \mu_v)^2}{\sigma_v^2})}\\
 & = (\frac{1}{2\pi\sqrt{|\Sigma|}})^ne^{-\frac{1}{2}\sum_{i = 1}^{n}(\frac{(u_i - \mu_u)^2}{\sigma_u^2} + \frac{(v_i - \mu_v)^2}{\sigma_v^2})}\\
 & = (\frac{1}{2\pi\sqrt{|\Sigma|}})^ne^{-\frac{1}{2}\sum_{i = 1}^{n}(\frac{(u_i - \bar{u})^2}{\sigma_u^2} + \frac{(v_i - \bar{v})^2}{\sigma_v^2} + \frac{(\bar{u} - \mu_u)^2}{\sigma_u^2} + \frac{(\bar{v} - \mu_v)^2}{\sigma_v^2})}\\
 & =  (\frac{1}{2\pi\sqrt{|\Sigma|}})^ne^{-\frac{n}{2}(\bar{\mathbf{z}} - \mu_{\mathbf{z}})\Sigma^{-1}(\bar{\mathbf{z}} - \mu_{\mathbf{z}})^T}\exp(-\frac{1}{2}\sum_{i = 1}^{n}tr(\begin{pmatrix}
 \frac{1}{\sigma_u^2} & 0\\
 0& \frac{1}{\sigma_v^2}
 \end{pmatrix}
 (\mathbf{w}_i - \bar{\mathbf{w}})^T(\mathbf{w}_i - \bar{\mathbf{w}})
 )\\
 &  =  (\frac{1}{2\pi\sqrt{|\Sigma|}})^ne^{-\frac{n}{2}(\bar{\mathbf{z}} - \mu_{\mathbf{z}})\Sigma^{-1}(\bar{\mathbf{z}} - \mu_{\mathbf{z}})^T}\exp(-\frac{1}{2}\sum_{i = 1}^{n}tr(\Sigma^{-1}(\mathbf{z}_i - \bar{\mathbf{z}})^T(\mathbf{z}_i - \bar{\mathbf{z}}))
 \end{split}&
 \end{flalign*}
 设\(\mathbf{S} = \sum_{i = 1}^{n}(\mathbf{z}_i - \bar{\mathbf{z}})^T(\mathbf{z}_i - \bar{\mathbf{z}})\)\\
 有\[L = (\frac{1}{2\pi\sqrt{|\Sigma|}})^ne^{-\frac{n}{2}(\bar{\mathbf{z}} - \mu_{\mathbf{z}})\Sigma^{-1}(\bar{\mathbf{z}} - \mu_{\mathbf{z}})^T}\exp(-\frac{1}{2}tr(\Sigma^{-1}\mathbf{S}))\]
 考虑到协方差矩阵的性质,对指数上第一项\( (\bar{\mathbf{z}} - \mu_{\mathbf{z}})\Sigma^{-1}(\bar{\mathbf{z}} - \mu_{\mathbf{z}})^T \),仅当\(\hat{\mu}_{\mathbf{Z}} = \bar{\mathbf{z}}\)时取得0,对应\(L\)为极大。\\
 再考虑\(\displaystyle L' = (\frac{1}{\sqrt{|\Sigma|}})^n\exp(-\frac{1}{2}tr(\Sigma^{-1}\mathbf{S}))\\
 \log L' = -\frac{n}{2}\log |\Sigma| - \frac{1}{2}tr(\Sigma^{-1}\mathbf{S}) = \frac{n}{2}\log |\Sigma^{-1}| - \frac{1}{2}tr(\Sigma^{-1}\mathbf{S})\\
 \text{对} \Sigma^{-1}\text{的每一分量求导,结果写为矩阵形式,有}\\
 \frac{\partial}{\partial \Sigma^{- 1}}\log L' = \frac{n}{2}\Sigma - \frac{1}{2}\mathbf{S}\\
 \text{该方程只有一个解,表明极值只有一个,并且 }|\Sigma| \to \infty\text{时} L' \to 0,\text{而当 } \Sigma = \frac{1}{n}\mathbf{S}\text{时,}L'\text{为正},\text{故该极值为极大值}
 \text{因此 } \hat{\Sigma} = \frac{1}{n}\mathbf{S}\)\\
 若将矩阵写为分量形式即与(a)问中得到的结果相同。
\end{document}
