\documentclass[11pt,a4paper]{ctexart}
\usepackage{amsmath,amssymb,amsthm}
\linespread{1.5}
\CTEXsetup[indent={0pt}]{subparagraph}
\newcommand{\normal}[2]{\frac{1}{\sqrt{2\pi}#2}e^{-(x - #1)^2/(2#2^2)}}
\newcommand{\norm}[1]{\frac{1}{\sqrt{2\pi}#1}e^{-x^2/(2#1^2)}}
\newcommand{\norms}[0]{\frac{1}{\sqrt{2\pi}}e^{-x^2/2}}
\newcommand{\dconverge}[0]{\overset{\mathcal{D}}{\to}}
\newcommand{\pconverge}[0]{\overset{\mathcal{P}}{\to}}
\title{第一次检测}
\author{基科32 曾柯又 2013012266}
\date{\vspace{-5ex}}
\begin{document}
\abovedisplayskip=5pt
\belowdisplayskip=5pt
\abovedisplayshortskip=0pt
\belowdisplayshortskip=0pt
\maketitle
\paragraph{1}
可令$r_n = n$\\
由中心极限定理$\sqrt{n}(\hat{\mu_n} - \mu) \overset{d}{\to} n(0,\sigma^2)$\\
设\(g(x) = \cos(x)\),有\( g'(0) = 0 \; g''(0) = -1\)\\
由二阶delta方法\[ n(\cos(\hat{\mu_n}) - \cos(\mu) ) \overset{d}{\to} -\frac{\sigma^2}{2}\chi^2_1 \]
即收敛到非退化随机变量
 \paragraph{2}
(a)先求出\(X_{(n+1)} , X_{(2n)}\)的联合概率分布,为简化记号,记\(U = X_{(n+1)},\;V = X_{(2n)}\),有:
 \[ f_{U,V}(u,v) = \frac{(2n + 1)!}{n!(2n - 2)!}f_X(u)f_X(v)F_X^n(u)[F_X(v) - F_X(u)]^{n - 2}[1 - F_X(v)]\]
 \(F_X(u) = \frac{1}{\theta},\;F_X(u) = \frac{u}{\theta}\quad 0 \leq u \leq v \leq \theta\)  可得:
 \begin{equation*}
\begin{split}
f_{U,V}(u,v) & = \frac{(2n + 1)!}{n!(2n - 2)!}(\frac{1}{\theta})^2(\frac{u}{\theta})^n(\frac{v - u}{\theta})^{n - 2}(1 -\frac{v}{\theta} )\\
& = \frac{(2n + 1)!}{n!(2n - 2)!\theta^{2n + 1}}u^n(v - u)^{n - 2}(\theta - v)  \qquad 0 \leq u \leq v \leq \theta 
\end{split}
 \end{equation*}
 设 \(S =  V + U ,\; T = V - U\) , 则:
 \begin{equation*}
 \begin{split}
 f_{S,T}(s,t) & = f_{U,V}(\frac{s - t}{2},\frac{s + t}{2})\Big|\frac{\partial(u,v)}{\partial(s,t)}\Big|\\
 & = \frac{(2n + 1)!}{n!(n-2)!\theta^{2n + 1}}(\frac{s - t}{2})^nt^{n - 2}(\theta - \frac{s + t}{2})\times\frac{1}{2}
 \end{split}
 \end{equation*}
 $s,t$的取值范围为: 
\(s \in [0, \theta]\) 时,\( t \in [0,s]\) ; \(s \in [\theta, 2\theta]\) 时,\( t \in [0,2\theta - s]\) ,因此,当\(s \in [0, \theta]\) 时:
\begin{equation*}
\begin{split}
f_S(s) & = \int_{0}^{s}f_{S,T}(s,t)\mathrm{d}t \\
&= \int_{0}^{s}\frac{(2n + 1)!s^{2n - 1}}{n!(n-2)!\theta^{2n + 2}2^{n + 1}}(1 - \frac{t}{s})^n(\frac{t}{s})^{n - 2}(\frac{2\theta}{s} - 1 -  \frac{t}{s})\mathrm{d}t\\
& = \int_{0}^{1}\frac{(2n + 1)!s^{2n}}{n!(n-2)!\theta^{2n + 1}2^{n + 2}}(1 - t)^nt^{n - 2}(\frac{2\theta}{s} - 1 -  t)\mathrm{d}t\\
& = \frac{(2n + 1)}{2^{n + 2}\theta}(4n(\frac{s}{\theta})^{2n - 1} - (3n - 1)(\frac{s}{\theta})^{2n})
\end{split}
\end{equation*}
当\(s \in [\theta, 2\theta]\) 时:
\begin{equation*}
\begin{split}
f_S(s) & = \int_{0}^{2\theta - s}f_{S,T}(s,t)\mathrm{d}t \\
& = ...
\end{split}
\end{equation*}
好像没有初等表达式
\paragraph{3}
可取\(r_n = \sqrt{n} ,\; \phi(x) = 2\arcsin(\sqrt{x})\)\\
由中心极限定理:
\[ \sqrt{n}(\hat{p_n} - p) \overset{d}{\to} n(0,p(1 - p))\]
而$\phi'(p) = \sqrt{\frac{1}{p(1 - p)}}$,且$p \in (0 , 1)$,故$\phi'(p) \neq 0 $由delta方法:
\[ \sqrt{n}(\phi(\hat{p_n}) - \phi(p)) \overset{d}{\to} n(0,1)\]
\paragraph{4}
$\displaystyle
P(n(1 - X_{(n)}) \leq t)  = P(X_{(n)} > 1 - \frac{t}{n}) = 1 - P(X_{(n)} < 1 - \frac{t}{n})\\$
而$\displaystyle P(X_{(n)} < 1 - \frac{t}{n}) = \prod_{\substack{i = 1}}^{\substack{n}} P(X_i < 1 - \frac{t}{n}) = (1 - \frac{t}{n})^n $\\
故$\displaystyle P(n(1 - X_{(n)}) \leq t) = 1 - (1 - \frac{t}{n})^n $ \\
 \(\Rightarrow \lim\limits_{n\to \infty}P(n(1 - X_{(n)}) \leq t) = 1 - e^{-t}\)\\
 即为指数分布 $\displaystyle n(1 - X_{(n)}) \overset{d}{\to} X, \quad f_X(x) = e^{-x}$
 \paragraph{5}
 由中心极限定理 
 \[ \sqrt{n}(\bar{X_n} - \mu) \overset{d}{\to} n(0,\sigma^2)\]
 记\(g(x) = x^2 ,\;g'(\mu) = 2 \mu,\; g''(\mu) = 2\)\\
 若\( \mu \neq 0\),由delta方法:
 \[ \sqrt{n}((\bar{X_n})^2 - \mu^2) \overset{d}{\to} n(0,4\sigma^2\mu^2)\]
 即可取\(c_n = \sqrt{n},\;A = \mu^2\)\\
 若\(\mu = 0\),由二阶delta方法:
 \[n((\bar{X_n})^2 - \mu^2) \overset{d}{\to} \sigma^2\chi^2_1\]
 即取\(c_n = n ,\; A = \mu^2\)
 \paragraph{6}(a)第一部分
 \(\sqrt{n}(\bar{X_n} - \mu) - \frac{1}{\sqrt{n}}\sum_{i = 1}^{n}(X_i - \mu) = 0\)\\
 而\begin{flalign*}
 \begin{split}
  \sqrt{n}&(S_n^2 -\sigma^2) - \frac{1}{\sqrt{n}}\sum_{i = 1}^{n}[(X_i - \mu)^2 - \sigma^2]\\
&  = \sqrt{n}(S_n^2 - \sigma^2) - \frac{1}{\sqrt{n}}\sum_{i = 1}^{n}[(X_i - \bar{X_n} + \bar{X_n} - \mu)^2 - \sigma^2]\\
 &  = -\sqrt{n}(\bar{X_n} - \mu)^2
 \end{split}
  \end{flalign*}
  而\(\sqrt{n}(\bar{X_n} - \mu) \overset{p}{\to} n(0,1)\),由连续映射定理:\[n(\bar{X_n} - \mu)^2 \overset{p}{\to} \chi_1^2\]故:\[\sqrt{n}(\bar{X_n} - \mu)^2 \overset{p}{\to}  0\]
  因此\[\sqrt{n}\begin{pmatrix}
  \bar{X_n} - \mu\\
  S_n^2 - \sigma^2
  \end{pmatrix} - \frac{1}{\sqrt{n}}\sum_{i = 1}^{n}\begin{pmatrix}
  X_i - \mu\\
  (X_i - \mu)^2 - \sigma^2
  \end{pmatrix} \overset{p}{\to} 0 \]
  即 \[\sqrt{n}\begin{pmatrix}
    \bar{X_n} - \mu\\
    S_n^2 - \sigma^2
    \end{pmatrix} = \frac{1}{\sqrt{n}}\sum_{i = 1}^{n}\begin{pmatrix}
    X_i - \mu\\
    (X_i - \mu)^2 - \sigma^2
    \end{pmatrix} + o_p(1)\]
    (b)记\(\mathcal{X}_i =\begin{pmatrix}
        X_i - \mu\\
        (X_i - \mu)^2 - \sigma^2
        \end{pmatrix} \) ,有\(E\mathcal{X}_i = 0,\;\mathrm{cov}(\mathcal{X}_i) = \boldsymbol{\Sigma}\),经过计算可得协方差矩阵\(\boldsymbol{\Sigma}\)
        \[ \begin{pmatrix}
        E(X_i - \mu)^2 & E(X_i - \mu)((X_i - \mu)^2 - \sigma^2)\\
        E(X_i - \mu)((X_i - \mu)^2 - \sigma^2) & E((X_i - \mu)^2 - \sigma^2)^2
        \end{pmatrix} = \begin{pmatrix}
        \sigma^2 & \sigma^3\gamma_1\\
        \sigma^3\gamma_1 & \sigma^4(\gamma_2 + 2)
        \end{pmatrix}\]
        因此\[\sqrt{n}(\bar{\mathcal{X}_n}) \overset{d}{\to} \mathcal{N}_2(0,\boldsymbol{\Sigma})\]
        由slusky定理\[\sqrt{n}\begin{pmatrix}
            \bar{X_n} - \mu\\
            S_n^2 - \sigma^2
            \end{pmatrix} \overset{d}{\to} \mathcal{N}_2(0,\boldsymbol{\Sigma})\]
        (c)记\(\boldsymbol{Y_n} = \begin{pmatrix}
                    \bar{X_n} \\
                    S_n^2
                    \end{pmatrix} ,\; \boldsymbol{\theta} = \begin{pmatrix}
                                        \mu \\
                                        \sigma^2
                                        \end{pmatrix}\),则
                                        \[ \sqrt{n}(\boldsymbol{Y_n} -\boldsymbol{\theta}) \overset{d}{\to}\mathcal{N}_2(0,\boldsymbol{\Sigma}) \]
            设函数\(\displaystyle g(x_1,x_2) = \frac{x_1}{\sqrt{x_2}}, \quad g'(\mu,\sigma^2) = (
            \frac{1}{\sigma},  -\frac{\mu}{2\sigma^3}
            ) ,\quad = g'^{T}\boldsymbol{\Sigma}g' =1 - \frac{\mu\gamma_1}{\sigma} + (\frac{\mu}{2\sigma})^2(\gamma_2 + 2) \),则由多元函数的delta方法可得
            \[\sqrt{n}(g(\boldsymbol{Y_n}) - g(\boldsymbol{\theta}) ) \overset{d}{\to} n(0,\tau^2)\]
            即\[ \sqrt{n}(\frac{\bar{X_n}}{S_n} - \frac{\mu}{\sigma}) \overset{d}{\to} n(0,\tau^2) \]
            其中 \( \tau^2 = 1 - \frac{\mu\gamma_1}{\sigma} + (\frac{\mu}{2\sigma})^2(\gamma_2 + 2) \)
            
\end{document}
