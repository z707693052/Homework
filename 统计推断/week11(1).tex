\documentclass[11pt,a4paper]{ctexart}
\usepackage{amsmath,amssymb,amsthm}
\usepackage{graphicx}
\DeclareGraphicsExtensions{.pdf,.png,.jpg,.eps}
\graphicspath{{/home/cengq/Pictures/}}
\linespread{1.9}
\CTEXsetup[indent={0pt}]{subparagraph}
\newcommand{\normal}[2]{\frac{1}{\sqrt{2\pi}#2}e^{-(x - #1)^2/(2#2^2)}}
\newcommand{\norm}[1]{\frac{1}{\sqrt{2\pi}#1}e^{-x^2/(2#1^2)}}
\newcommand{\norms}[0]{\frac{1}{\sqrt{2\pi}}e^{-x^2/2}}
\newcommand{\dconverge}[0]{\overset{\mathcal{D}}{\to}}
\newcommand{\pconverge}[0]{\overset{\mathcal{P}}{\to}}
\newcommand{\dd}[0]{\mathrm{d}}
\title{\vspace{-5ex}}
\author{基科32 曾柯又 2013012266}
\date{\vspace{-5ex}}
\begin{document}
\abovedisplayskip=5pt
\belowdisplayskip=5pt
\abovedisplayshortskip=0pt
\belowdisplayshortskip=0pt
\maketitle
\paragraph{7.44}
先证明\(\bar{X}^2 - \frac{1}{n}\)是一个无偏估计\\
\(\displaystyle \because\; \bar{X} \sim n(\theta,\frac{1}{n})\\
\therefore \;\sqrt{n}(\bar{X} - \theta)\sim(0,1) \Rightarrow n(\bar{X} - \theta)^2 \sim \chi^2_1\\
\therefore nE(\bar{X} - \theta)^2 = nE(\bar{X}^2 - \theta^2) = 1\\
\therefore E(\bar{X}^2 - \frac{1}{n}) = \theta^2\)
由指数分布族的性质可知\(\bar{X}\)是\(\theta\)的完全统计量,且由前面的例子可知\(\bar{X}\)是充分统计量,因此\(\bar{X}^2 - \frac{1}{n}\)是\(\theta\)的最佳无偏估计。
\begin{flalign*}
\begin{split}
Var(\bar{X}^2 - \frac{1}{n}) & = E(\bar{X}^2 - \frac{1}{n} - \theta^2)^2\\
& = E[( \bar{X} - \theta) (\bar{X} + \theta) - \frac{1}{n}]^2\\
& = E(\bar{X} + \theta)^2(\bar{X} - \theta)^2 - \frac{2}{n}E(\bar{X} - \theta)(\bar{X} + \theta) + \frac{1}{n^2}\\
& = \frac{1}{n}E[2(\bar{X} + \theta)(\bar{X} - \theta) + (\bar{X} + \theta)^2] - \frac{2}{n^2} + \frac{1}{n^2}\\
& = \frac{1}{n}(\frac{2}{n} + E(\bar{X}^2 + 2\bar{X}\theta + \theta^2)) - \frac{2}{n^2} + \frac{1}{n^2}\\
& = \frac{1}{n}(4\theta^2 + \frac{3}{n}) - \frac{2}{n^2} + \frac{1}{n^2}\\
& = \frac{4\theta^2}{n} + \frac{2}{n^2}
\end{split}&
\end{flalign*}
而由Cramer-Rao不等式给出的下界为:
\begin{flalign*}
\begin{split}
\frac{(2\theta)^2}{-nE(\frac{\partial^2}{\partial \theta^2}\log f)}
 & = \frac{(2\theta)^2}{-nE[\frac{\partial^2}{\partial \theta^2}(-\frac{1}{2}\log(2\pi) - \frac{1}{2}(x - \theta)^2)]}\\
 & = \frac{4\theta^2}{n}
\end{split}&
\end{flalign*}
即\(Var(\bar{X} - \frac{1}{n^2}) > \frac{4\theta^2}{n}\)
\paragraph{7.45}
\subparagraph{(a)}
\begin{flalign*}
\begin{split}
MSE(aS^2) & = E(aS^2 - \sigma^2)^2\\
& = E(a(S^2 - \sigma^2) + (a -1)\sigma^2)^2\\
& = E[a^2(S^2 - \sigma^2)^2 + (a - 1)^2\sigma^4 + 2a(S^2 - \sigma^2)(a - 1)\sigma^2]\\
& = a^2VarS^2 + (a - 1)^2\sigma^2
\end{split}&
\end{flalign*}
\subparagraph{(b)}
首先容易证明\(\displaystyle S^2 = \frac{1}{2n(n - 1)}\sum_{i = 1}^{n}\sum_{j = 1}^{n}(X_i - X_j)^2\)\\
记\(Z_ i  = X_i - \mu ,\; \text{有} EZ_i = 0 , \;EZ_i^2 = \sigma^2,\; EZ_i^4 = \kappa\)\\
有\(\displaystyle S^2 = \frac{1}{2n(n - 1)}\sum_{i = 1}^{n}\sum_{j = 1}^{n}(Z_i - Z_j)^2\)\\
\(\displaystyle S^4 = (\frac{1}{2n(n - 1)})^2\sum_{i = 1}^{n}\sum_{j = 1}^{n}\sum_{k = 1}^{n}\sum_{l = 1}^{n}(Z_i -Z_j)^2(Z_k - Z_l)^2\)\\
根据\(i,j,k,l\)的不同取值,平均值不同,先分情况讨论\\
有\(n(n - 1)(n - 2)(n - 3)\)对满足\(i,j,k,l\)互不相同\\
\(E(Z_1 - Z_2)^2(Z_3-Z_4)^2 = (2\sigma^2)^2 = 4\sigma^4\)\\
有\(2n(n - 1)\)对满足\(i \neq j \text{ 且 } i = k ,\;j = l\text{或} i = l ,\;j = k\)\\
\(E(Z_1 - Z_2)^4 = E(Z_1^4 - 4Z_1^3Z_2 + 6Z_1^2Z_2^2 - 4Z_1Z_2^3 + Z_2^4) = 2\kappa + 6\sigma^4\)\\
有\(4n(n - 1)(n - 2)\)对满足\(i,j,k,l\)中仅有一对相等且\(i \neq j,\; k \neq l\)\\
\(E(Z_1 - Z_2)^2(Z_1 - Z_3)^2 = E(Z_1^2 + Z_2^2 - 2Z_1Z_2)(Z_1^2 + Z_3^2 - 2Z_1Z_3) = \kappa + 3\sigma^4\)\\
因此\begin{flalign*}
\begin{split}
ES^4 & = \left[\frac{1}{2n(n - 1)}\right]^2[4n(n - 1)(n - 2)(n - 3)\sigma^4 + 2n(n - 1))(2\kappa + 6\sigma^4) \\
& + 4n(n - 1)(n - 2)(\kappa + 3\sigma^4)]\\
& = \left[\frac{1}{2n(n - 1)}\right]^2\left[4n(n - 1)^2\kappa + 4\sigma^4n(n - 1)(n^2 - 2n + 3)\right]
\end{split}&
\end{flalign*}
\begin{flalign*}
\begin{split}
VarS^2 &= ES^4 - (ES^2)^2\\
& = \frac{\kappa}{n} + \frac{n^2 - 2n + 3}{n(n - 1)}\sigma^4 - \sigma^4\\
& = \frac{\kappa}{n} + \frac{(3 - n)}{n(n - 1)}\sigma^4\\
& = \frac{1}{n}(\kappa - \frac{n - 3}{n - 1}\sigma^4)
\end{split}
\end{flalign*}
\subparagraph{(c)}
若\(X \sim n(\mu,\sigma^2)\)
\begin{flalign*}
\begin{split}
E(X - \mu)^4 & = E(X - \mu)^3(X - \mu)\\
& = \sigma^2E3(X - \mu)^2\\
& = 3\sigma^4
\end{split}&
\end{flalign*}
\(\displaystyle \therefore \kappa = 3\sigma^4 \Rightarrow Var(S^2) = \frac{1}{n}(\kappa - \frac{n - 3}{n - 1}\sigma^4) = \frac{2\sigma^4}{n - 1}\\
MSE(aS^2) = a^2VarS^2 + (a - 1)^2\sigma^4 = a^2\frac{2\sigma^4}{n - 1} + (a - 1)^2\sigma^4\\
\text{对}a\text{求导可得 }a = \frac{n - 1}{n + 1}, \; \text{而该二次函数的极值为极小值,故MSE取极}\\\text{小值对应的}aS^2\text{为}
\frac{n - 1}{n + 1}S^2\\
\)
\subparagraph{(d)}
\(\displaystyle MSE(aS^2) = a^2\frac{1}{n}(\kappa - \frac{n - 3}{n - 1}\sigma^4) + (a - 1)^2\sigma^4
\)
对\(a\)求导可得\\
\(\displaystyle a\frac{1}{n}(\frac{\kappa}{\sigma^4} - \frac{n - 3}{n - 1}) + (a - 1) = 0\\
\Rightarrow a = \frac{n - 1}{(n + 1)+ \frac{(n - 1)(\frac{\kappa}{\sigma^4} - 3)}{n}}
\)\\
由于\(\displaystyle Var(S^2) > 0\; \Rightarrow \; Var(S^2) + \sigma^4 > 0\)\\
a对应二次函数的2次幂系数大于0,故该极值恰对应极小值\\
书中的公式应该有错
\subparagraph{(e)}
若 \(\displaystyle \kappa > 3\sigma^4,\;\frac{(n - 1)(\frac{\kappa}{\sigma^4} - 3)}{n} > 0 \Rightarrow a < \frac{n - 1}{n + 1}\\
\text{若 } \kappa < 3\sigma^4,\; \frac{(n - 1)(\frac{\kappa}{\sigma^4} - 3)}{n} < 0 \Rightarrow a > \frac{n - 1}{n + 1}\\
\)
另一方面\(\displaystyle VarS^2 > 0 \Rightarrow \kappa > \frac{n - 3}{n - 1}\sigma^4\\
\therefore \frac{(n - 1)(\frac{\kappa}{\sigma^4} - 3)}{n} > -\frac{2}{n - 1} > -2\\
 a = \frac{n - 1}{(n + 1)+ \frac{(n - 1)(\frac{\kappa}{\sigma^4} - 3)}{n}} <  \frac{n - 1}{(n + 1)+ -2} = 1\)
\end{document}
